
\documentclass[runningheads]{cl2emult}

\usepackage{makeidx}  % allows index generation
\usepackage{graphicx} % standard LaTeX graphics tool
                      % for including eps-figure files
\usepackage{subeqnar} % subnumbers individual equations
                      % within an array
\usepackage{multicol} % used for the two-column index
\usepackage{cropmark} % cropmarks for pages without
                      % pagenumbers
\usepackage{math}     % placeholder for figures
\usepackage{amssymb}     % placeholder for figures
\usepackage{subfigure} % subfigures in one figure

%\usepackage{algorithm2e}
\usepackage[ruled,algonl]{algorithm2e}
%\setlength{\algomargin}{2.1em}
\dontprintsemicolon
\SetInd{0.5em}{1em}
\SetKwFor{Forall}{forall}{do}{od}
\SetKwFor{WhileDo}{while}{do}{od}
\SetKw{Let}{let}

\makeindex            % used for the subject index
                      % please use the style sprmidx.sty with
                      % your makeindex program

%upright Greek letters (example below: upright "mu")
\newcommand{\euler}[1]{{\usefont{U}{eur}{m}{n}#1}}
\newcommand{\eulerbold}[1]{{\usefont{U}{eur}{b}{n}#1}}
\newcommand{\umu}{\mbox{\euler{\char22}}}
\newcommand{\umub}{\mbox{\eulerbold{\char22}}}
\newcommand{\url}[1]{{\small{\tt #1}}}

%%%%%%%%%%%%%%%%%%%%%%%%%%%%%%%%%%%%%%%%%%%%%%%%%%%%%%%%%%%%%

%OPTIONAL%%%%%%%%%%%%%%%%%%%%%%%%%%%%%%%%%%%%%%%%%%%%%%%%%%%%
%
%\usepackage{amstex}   % useful for coding complex math
%\mathindent\parindent % needed in case "Amstex" is used
%
%%%%%%%%%%%%%%%%%%%%%%%%%%%%%%%%%%%%%%%%%%%%%%%%%%%%%%%%%%%%%

%AUTHOR_STYLES_AND_DEFINITIONS%%%%%%%%%%%%%%%%%%%%%%%%%%%%%%%
%
%Please reduce your own definitions and macros to an absolute
%minimum since otherwise the editor will find it rather
%strenuous to compile all individual contributions to a
%single book file
%
%%%%%%%%%%%%%%%%%%%%%%%%%%%%%%%%%%%%%%%%%%%%%%%%%%%%%%%%%%%%%

\begin{document}
%
\title*{The WilmaScope 3D Graph Visualisation System}
%
%
\toctitle{WilmaScope}
% allows explicit linebreak for the table of content
%
%
\titlerunning{WilmaScope}
% allows abbreviation of title, if the full title is too long
% to fit in the running head
%
\author{Tim Dwyer\inst{1}
\and Peter Eckersley\inst{2}
}
%
\authorrunning{Tim Dwyer and Peter Eckersley}
% if there are more than two authors,
% please abbreviate author list for running head
%
%
\institute{School of Information Technologies,
     Madsen Building F09,
     University of Sydney,
     NSW 2006,
     Australia.
		 E-mail: \url{dwyer@cs.usyd.edu.au}
\and Department of Computer Science \& Software Engineering,
     University of Melbourne,
     Victoria 3010,
     Australia.
		 E-mail: \url{pde@cs.mu.oz.au}}

\maketitle              % typesets the title of the contribution

%\begin{abstract}
%The abstract\index{abstract} should summarize the contents of the paper
%in at least 70 and at most 150 words; neither too long
%nor too short but to the point!
%\end{abstract}

\section{Introduction}\label{sec:intro}

\subsection{Witty Survey Subsection}
{\em survey previous stuff.}

\subsection{Debonair aims for the sultry Wilma}
\label{motivation}

Despite, or perhaps because of, the extensive research literature on 
graph drawing techniques, there is a lack of general-purpose visualisation
systems, particularly for application to 3-dimensional embeddings.

\paragraph{}

Graph drawing problems are, of course, exsitent within an enourmous range of
fields; a key motivation for creating a general purpose 3D visualisation
system is to provide easy-to-use components which can be employed by future
software across these application domains.

\paragraph{}

Within the graph drawing community itself, a system may also aspire to 
streamline, elucidate and beautify the work of algorithm design, comparison
and optimisation.

\section{Design}\label{sec:design}
\subsection{Architecture}
Conceived as a research project with open ended goals, Wilma was
designed to be as flexible and extensible as possible.  The intention
was to allow different components, such as new visual representations
for graph elements, interfaces (either graphical user interfaces or
remote programming interfaces) and different 
layout algorithm implementations to be added or removed easily.
Therefore the design needed to decouple these components as much as
possible so that altering one component would have minimal impact on
the other components.

\paragraph{}

To achieve this we began by exploring the well known Model-View-Controller
architecture (MVC) \cite{gangoffour}.  In an MVC architecture the Model is the core
of the application, the data and algorithms that automatically modify
the data.  The View is a user interface which displays information
about the model to the user.  The Controller is a separate user
interface that provides methods for the user to manipulate the
application, ie to control the Model.  Each components reference to
the other components is via a carefully defined interface which should
not require change one of the components is modified in some way.  The
standard data flow diagram for the MVC architecture is shown in Figure
\ref{fig:mvc}.

\paragraph{}

In our system the Model is further broken down into two components:
the Graph data structure itself, capable of representing the structure
and state of the graph including its arrangement in space, and
the Layout Engine, which is an abstraction of the basic methods required for
an implementation of any layout algorithm that will change the graphs
arrangement in space.

\paragraph{}

A further requirement for our system is that the graph data model and
the layout engine should able to be controlled either interactively by
the user, by loading data from a file or even by remote
procedure call from a program running in another process or possibly on
another machine.  Therefore the Controller component was also broken
down into several components.  A "bridge" layer provides a common
programmatic interface to the Model.  The methods provided by this
bridge layer can then be called by either the GUI interface component,
a component providing a CORBA interface for RPC or a component which
can load and save files to an XML format.

\paragraph{}

Figure \ref{fig:arch} shows a collaboration diagram for this overall
architecture. 

\subsection{Layout Algorithms}
\subsubsection{Force Directed}
\subsubsection{Multiscale Force Directed}

\subsubsection{Fast Simulated Annealing}

As we stated in Section~\ref{motivation}, a major design goal for Wilma
was to provide a platform within which algorithmic experimentation could be
conducted elegantly and efficiently.

\paragraph{}

Force-directed layout algorithms are of course the method of choice for a
wide range of graph visualisation problems.  One variation on these
algorithms, advocated by Fruchtermann \&
Rheingold~\cite{fruchtermann90force-directed} and Davidson {\em et.
al.}~\cite{davidson01noise}, attempts to avoid calculation of repulsion force
vectors, which is $O(n^2)$ in the number of verticies:  

\begin{equation}
\vec{R}_i \equiv \sum_{j \in G, j \neq i} \vec{r}_{i,j}
\end{equation}

\noindent Where $\vec{r}_{i,j} = \vec{r}_i - \vec{r}_j$ is the displacement
of node $i$ relative to $j$.

\paragraph{}

To avoid this expense, it is possible to approximate scalar energy potential
values, caching this information at grid points across space.  Iterated
updates to the potential well then cost only $O(|V|)$ time per iteration.  On
the downside, the cache array itself takes $O(v \centerdot d)$ memory, where
$v$ is the spatial volume of the embedded graph and $d$ is the density of
stored potential information.

\paragraph{}

Cowling~\cite{cowling02fast} has implemented a fast layout engine for
WilmaScope which employs simulated annealing with potential energy caching
to achieve linear-time graph embeddings.  Cowling was able to demonstrate
that the algorithms of \cite{davidson01noise} do appear to achieve linear
time results, but that when the grid side $v \centerdot d$ must be adjusted to
prevent folding in large graphs, space and time complexity are
no longer linear.

\paragraph{}

Cowling's work serves to demonstrate the utility of Wilma as a platform; with
the WilmaScope rendering and navigation system, and the availability of
convenient data structures for input and dataset management, algorithmic
experiments can be performed at optimal speeds.


\paragraph{}

\subsubsection{DOT: heirarchical layout}
\subsection{User Interfaces}
\subsection{Programming Interfaces}
\em{API/CORBA interface/etc...}

\section{The Importance of being Wilma -- Results}
\begin{itemize}
\item pretty pictures
\item applications
\item self-congratulatory monologues
\end{itemize}

\section{Conclusions}
\begin{itemize}
\item megalomaniacal rants
\item further work
\item witty closing line
\end{itemize}

%\begin{thebibliography}{7}
%%
%\addcontentsline{toc}{section}{References}
%
%\bibitem{CHT93} Cai, J., Han, X., Tarjan, R.~E.\ (1993)
%An algorithm for the maximal planar subgraph
%problem.
%SIAM J.\ Comput.\ {\bf22}, 1142--1164
%
%\bibitem{DETT99} Di Battista, G., Eades, P., Tamassia, R., Tollis,
%  I.~G.\ (1999)
%Graph Drawing: Algorithms for the visualization of graphs.
%Prentice Hall, New Jersey
%
%\bibitem{Dji95} Djidjev, H.~N.\ (1995)
%A linear algorithm for the maximal planar subgraph problem.
%Proc.\ 4th Workshop Algorithms Data Struct., 
%Lecture Notes in Computer Science, Springer Verlag
%
%\bibitem{Eul1750} Euler, L.\ (1750)
%Demonstratio nonnullarum insignium proprietatum quibus solida hedris
%planis inclusa sunt praedita.
%Novi Comm.\ Acad.\ Sci.\ Imp.\ Petropol.\ {\bf4} (1752-3, published
%1758), 140--160, also: Opera Omnia (1) {\bf26}, 94--108
%
%\bibitem{KW01} Kaufmann, M., Wagner, D. (eds.) (2001)
%Drawing Graphs: Methods and Models.
%Lecture Notes in Computer Science 2025, Springer Verlag
%
%\bibitem{TDB88} Tamassia, R., Di Battista, G., Batini, C.\ (1988)
%Automatic graph drawing and readability of diagrams.
%IEEE Transactions on Systems, Man, and Cybernetics {\bf18}, 61--79
%
%\end{thebibliography}

\bibliographystyle{plain}
\newpage
\addcontentsline{toc}{section}{References}
\bibliography{refs}

%
%INDEX%%%%%%%%%%%%%%%%%%%%%%%%%%%%%%%%%%%%%%%%%%%%%%%%%%%%%%%%%%%%%%%
%\clearpage
%\addcontentsline{toc}{section}{Index}
%\flushbottom
%\printindex
%%%%%%%%%%%%%%%%%%%%%%%%%%%%%%%%%%%%%%%%%%%%%%%%%%%%%%%%%%%%%%%%%%%%%%

\end{document}
