
\documentclass[runningheads]{cl2emult}

\usepackage{makeidx}  % allows index generation
\usepackage{graphicx} % standard LaTeX graphics tool
                      % for including eps-figure files
\usepackage{subeqnar} % subnumbers individual equations
                      % within an array
\usepackage{multicol} % used for the two-column index
\usepackage{cropmark} % cropmarks for pages without
                      % pagenumbers
\usepackage{math}     % placeholder for figures
\usepackage{amssymb}     % placeholder for figures
\usepackage{subfigure} % subfigures in one figure

%\usepackage{algorithm2e}
\usepackage[ruled,algonl]{algorithm2e}
%\setlength{\algomargin}{2.1em}
\dontprintsemicolon
\SetInd{0.5em}{1em}
\SetKwFor{Forall}{forall}{do}{od}
\SetKwFor{WhileDo}{while}{do}{od}
\SetKw{Let}{let}

\makeindex            % used for the subject index
                      % please use the style sprmidx.sty with
                      % your makeindex program

%upright Greek letters (example below: upright "mu")
\newcommand{\euler}[1]{{\usefont{U}{eur}{m}{n}#1}}
\newcommand{\eulerbold}[1]{{\usefont{U}{eur}{b}{n}#1}}
\newcommand{\umu}{\mbox{\euler{\char22}}}
\newcommand{\umub}{\mbox{\eulerbold{\char22}}}
\newcommand{\url}[1]{{\small{\tt #1}}}

%%%%%%%%%%%%%%%%%%%%%%%%%%%%%%%%%%%%%%%%%%%%%%%%%%%%%%%%%%%%%

%OPTIONAL%%%%%%%%%%%%%%%%%%%%%%%%%%%%%%%%%%%%%%%%%%%%%%%%%%%%
%
%\usepackage{amstex}   % useful for coding complex math
%\mathindent\parindent % needed in case "Amstex" is used
%
%%%%%%%%%%%%%%%%%%%%%%%%%%%%%%%%%%%%%%%%%%%%%%%%%%%%%%%%%%%%%

%AUTHOR_STYLES_AND_DEFINITIONS%%%%%%%%%%%%%%%%%%%%%%%%%%%%%%%
%
%Please reduce your own definitions and macros to an absolute
%minimum since otherwise the editor will find it rather
%strenuous to compile all individual contributions to a
%single book file
%
%%%%%%%%%%%%%%%%%%%%%%%%%%%%%%%%%%%%%%%%%%%%%%%%%%%%%%%%%%%%%

\begin{document}
%
\title*{The WilmaScope 3D Graph Visualisation System}
%
%
\toctitle{WilmaScope}
% allows explicit linebreak for the table of content
%
%
\titlerunning{WilmaScope}
% allows abbreviation of title, if the full title is too long
% to fit in the running head
%
\author{Tim Dwyer\inst{1}
\and Peter Eckersley\inst{2}
}
%
\authorrunning{Tim Dwyer and Peter Eckersley}
% if there are more than two authors,
% please abbreviate author list for running head
%
%
\institute{School of Information Technologies,
     Madsen Building F09,
     University of Sydney,
     NSW 2006,
     Australia
\and Department of Computer Science,
     University of Melbourne,
     Victoria 3010,
     Australia}

\maketitle              % typesets the title of the contribution

%\begin{abstract}
%The abstract\index{abstract} should summarize the contents of the paper
%in at least 70 and at most 150 words; neither too long
%nor too short but to the point!
%\end{abstract}

\section{Introduction}\label{sec:intro}

\subsection{Witty Survey Subsection}
\em{survey previous stuff.}

\subsection{Debonair aims for the sultry Wilma}
\label{motivation}

\em{motivation for wilma.}
%
\section{Design}\label{sec:design}
\subsection{Architecture}
\subsection{Layout Algorithms}
\subsubsection{Force Directed}
\subsubsection{Multiscale Force Directed}

\subsubsection{Fast Simulated Annealing}

As we stated in Section~\ref{motivation}, a major design goal for Wilma
was to provide a platform within which algorithmic experimentation could be
conducted elegantly and efficiently.

\paragraph{}

Force-directed layout algorithms are of course the method of choice for a
wide range of graph visualisation problems.  One variation on these
algorithms, advocated by Fruchtermann \&
Rheingold~\cite{fruchtermann90force-directed} and Davidson {\em et.
al.}~\cite{davidson01noise}, attempts to avoid calculation of repulsion force
vectors, which is $O(n^2)$ in the number of verticies:  

\begin{equation}
\vec{R}_i \equiv \sum_{j \in G, j \neq i} \vec{r}_{i,j}
\end{equation}

\noindent Where $\vec{r}_{i,j} = \vec{r}_i - \vec{r}_j$ is the displacement
of node $i$ relative to $j$.

\paragraph{}

To avoid this expense, it is possible to approximate scalar energy potential
values, caching this information at grid points across space.  Iterated
updates to the potential well then cost only $O(|V|)$ time per iteration.  On
the downside, the cache array itself takes $O(v \centerdot d)$ memory, where
$v$ is the spatial volume of the embedded graph and $d$ is the density of
stored potential information.

\paragraph{}

Cowling~\cite{cowling02fast} has implemented a fast layout engine for
WilmaScope which employs simulated annealing with potential energy caching
to achieve linear-time graph embeddings.  Cowling was able to demonstrate
that the algorithms of \cite{davidson01noise} do appear to achieve linear
time results, but that when the grid side $v \centerdot d$ must be adjusted to
prevent folding in large graphs, space and time complexity are
no longer linear.

\paragraph{}

Cowling's work serves to demonstrate the utility of Wilma as a platform; with
the WilmaScope rendering and navigation system, and the availability of
convenient data structures for input and dataset management, algorithmic
experiments can be performed at optimal speeds.


\paragraph{}

\subsubsection{DOT: heirarchical layout}
\subsection{User Interfaces}
\subsection{Programming Interfaces}
\em{API/CORBA interface/etc...}

\section{The Importance of being Wilma -- Results}
\begin{itemize}
\item pretty pictures
\item applications
\item self-congratulatory monologues
\end{itemize}

\section{Conclusions}
\begin{itemize}
\item megalomaniacal rants
\item further work
\item witty closing line
\end{itemize}

%\begin{thebibliography}{7}
%%
%\addcontentsline{toc}{section}{References}
%
%\bibitem{CHT93} Cai, J., Han, X., Tarjan, R.~E.\ (1993)
%An algorithm for the maximal planar subgraph
%problem.
%SIAM J.\ Comput.\ {\bf22}, 1142--1164
%
%\bibitem{DETT99} Di Battista, G., Eades, P., Tamassia, R., Tollis,
%  I.~G.\ (1999)
%Graph Drawing: Algorithms for the visualization of graphs.
%Prentice Hall, New Jersey
%
%\bibitem{Dji95} Djidjev, H.~N.\ (1995)
%A linear algorithm for the maximal planar subgraph problem.
%Proc.\ 4th Workshop Algorithms Data Struct., 
%Lecture Notes in Computer Science, Springer Verlag
%
%\bibitem{Eul1750} Euler, L.\ (1750)
%Demonstratio nonnullarum insignium proprietatum quibus solida hedris
%planis inclusa sunt praedita.
%Novi Comm.\ Acad.\ Sci.\ Imp.\ Petropol.\ {\bf4} (1752-3, published
%1758), 140--160, also: Opera Omnia (1) {\bf26}, 94--108
%
%\bibitem{KW01} Kaufmann, M., Wagner, D. (eds.) (2001)
%Drawing Graphs: Methods and Models.
%Lecture Notes in Computer Science 2025, Springer Verlag
%
%\bibitem{TDB88} Tamassia, R., Di Battista, G., Batini, C.\ (1988)
%Automatic graph drawing and readability of diagrams.
%IEEE Transactions on Systems, Man, and Cybernetics {\bf18}, 61--79
%
%\end{thebibliography}

\bibliographystyle{plain}
\newpage
\addcontentsline{toc}{section}{References}
\bibliography{refs}

%
%INDEX%%%%%%%%%%%%%%%%%%%%%%%%%%%%%%%%%%%%%%%%%%%%%%%%%%%%%%%%%%%%%%%
%\clearpage
%\addcontentsline{toc}{section}{Index}
%\flushbottom
%\printindex
%%%%%%%%%%%%%%%%%%%%%%%%%%%%%%%%%%%%%%%%%%%%%%%%%%%%%%%%%%%%%%%%%%%%%%

\end{document}
